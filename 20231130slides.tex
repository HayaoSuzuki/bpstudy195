\documentclass[aspectratio=169,dvipdfmx,12pt,notheorems]{beamer}
%%%% 和文用 %%%%%
\usepackage{bxdpx-beamer}
\usepackage{pxjahyper}
\usepackage{minijs}%和文用
\renewcommand{\kanjifamilydefault}{\gtdefault}%和文用

%%%% スライドの見た目 %%%%%
\usetheme{Madrid}
\usefonttheme{professionalfonts}
\setbeamertemplate{frametitle}[default][center]
\setbeamertemplate{navigation symbols}{}
\setbeamercovered{transparent}%好みに応じてどうぞ)
\setbeamertemplate{blocks}[rounded]
\useinnertheme{circles}
\setbeamertemplate{footline}[page number]
\setbeamerfont{footline}{size=\normalsize,series=\bfseries}
\setbeamercolor{footline}{fg=black,bg=black}
%%%%

%%%% 定義環境 %%%%%
\usepackage{amsmath,amssymb}
\usepackage{amsthm}
\theoremstyle{definition}
\newtheorem{theorem}{定理}
\newtheorem{definition}{定義}
\newtheorem{proposition}{命題}
\newtheorem{lemma}{補題}
\newtheorem{corollary}{系}
\newtheorem{conjecture}{予想}
\newtheorem*{remark}{Remark}
\renewcommand{\proofname}{}
%%%%%%%%%

%%%%% フォント基本設定 %%%%%
\usepackage[T1]{fontenc}%8bit フォント
\usepackage{textcomp}%欧文フォントの追加
\usepackage[utf8]{inputenc}%文字コードをUTF-8
\usepackage[deluxe]{otf}%otfパッケージ
\usepackage{lxfonts}%数式・英文ローマン体を Lxfont にする
\usepackage{bm}%数式太字
%%%%%%%%%%

%%%%% PythonTeX %%%%%
\usepackage[makestderr]{pythontex}
\restartpythontexsession{\thesection}

\usepackage{ulem}
 
\title{Python Distilled 試飲会}
\author[Hayao]{Hayao Suzuki}
\institute[BPStudy \#195]{BPStudy \#195}
\date{November 30, 2023}

\begin{document}

\begin{frame}[plain]\frametitle{}
\titlepage %表紙
\end{frame}


\section{Introduction}

\begin{frame}\frametitle{自己紹介}

\begin{block}{お前誰よ}
\begin{description}
\item[Name] Hayao Suzuki(鈴木 駿)
\item[\xout{Twitter} X] \href{https://twitter.com/CardinalXaro}{@CardinalXaro}
\item[Work] Software Developer @ BeProud Inc.
\end{description}
\end{block}

\begin{center}
\begin{itemize}
\item 株式会社ビープラウド \includegraphics[width=3cm]{bplogo.png} 
\begin{itemize}
\item IT勉強会支援プラットフォーム \includegraphics[width=2cm]{connpass_logo_1.png}
\item Python独学プラットフォーム \includegraphics[width=1cm]{pyq_logo_color.png} 
\item システム開発ドキュメントサービス \includegraphics[width=3cm]{tracery.png} 
\end{itemize}
\end{itemize}
\end{center}

\end{frame}

\begin{frame}\frametitle{自己紹介}

\begin{block}{発表したトーク(抜粋)}
\begin{itemize}
\item \structure{SymPyによる数式処理}(PyCon JP 2018)
\item \structure{インメモリーストリーム活用術}(PyCon JP 2020)
\item \structure{組み込み関数powの知られざる進化}(PyCon JP 2021)
\item \structure{Let's implement useless Python objects}(PyCon JP 2023)
\end{itemize}
\end{block}
\url{https://xaro.hatenablog.jp/}に一覧があります。
\end{frame}

\begin{frame}\frametitle{自己紹介}

\begin{block}{翻訳した本}
\begin{itemize}
\item \structure{Python Distilled}(O'Reilly Japan) \structure{本日の主役} 
\end{itemize}
\end{block}

\begin{block}{監訳した本}
\begin{itemize}
\item \structure{入門 Python 3 第2版}(O'Reilly Japan) 
\item \structure{ロバストPython}(O'Reilly Japan) 
\end{itemize}
\end{block}

\end{frame}

\begin{frame}\frametitle{今日のテーマ}

\begin{center}
\Huge{Python Distilled}
\end{center}

\end{frame}

\begin{frame}\frametitle{今日のテーマ}

\begin{center}
\includegraphics[width=5cm]{picture_large978-4-8144-0046-1.jpeg}
\end{center}

\end{frame}

\begin{frame}\frametitle{今日のテーマ}

\begin{center}
\includegraphics[width=5cm]{original.jpg}
\end{center}

\end{frame}

























\end{document}