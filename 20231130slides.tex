\documentclass[aspectratio=169,dvipdfmx,12pt,notheorems]{beamer}
%%%% 和文用 %%%%%
\usepackage{bxdpx-beamer}
\usepackage{pxjahyper}
\usepackage{minijs}%和文用
\renewcommand{\kanjifamilydefault}{\gtdefault}%和文用

%%%% スライドの見た目 %%%%%
\usetheme{Madrid}
\usefonttheme{professionalfonts}
\setbeamertemplate{frametitle}[default][center]
\setbeamertemplate{navigation symbols}{}
\setbeamercovered{transparent}%好みに応じてどうぞ)
\setbeamertemplate{blocks}[rounded]
\useinnertheme{circles}
\setbeamertemplate{footline}[page number]
\setbeamerfont{footline}{size=\normalsize,series=\bfseries}
\setbeamercolor{footline}{fg=black,bg=black}
%%%%

%%%% 定義環境 %%%%%
\usepackage{amsmath,amssymb}
\usepackage{amsthm}
\theoremstyle{definition}
\newtheorem{theorem}{定理}
\newtheorem{definition}{定義}
\newtheorem{proposition}{命題}
\newtheorem{lemma}{補題}
\newtheorem{corollary}{系}
\newtheorem{conjecture}{予想}
\newtheorem*{remark}{Remark}
\renewcommand{\proofname}{}
%%%%%%%%%

%%%%% フォント基本設定 %%%%%
\usepackage[T1]{fontenc}%8bit フォント
\usepackage{textcomp}%欧文フォントの追加
\usepackage[utf8]{inputenc}%文字コードをUTF-8
\usepackage[deluxe]{otf}%otfパッケージ
\usepackage{lxfonts}%数式・英文ローマン体を Lxfont にする
\usepackage{bm}%数式太字
%%%%%%%%%%

%%%%% PythonTeX %%%%%
\usepackage[makestderr]{pythontex}
\restartpythontexsession{\thesection}

\usepackage{ulem}
 
\title{Python Distilled 試飲会}
\author[Hayao]{Hayao Suzuki}
\institute[BPStudy \#195]{BPStudy \#195}
\date{November 30, 2023}

\begin{document}

\begin{frame}[plain]\frametitle{}
\titlepage %表紙
\end{frame}


\section{Introduction}

\begin{frame}\frametitle{自己紹介}

\begin{block}{お前誰よ}
\begin{description}
\item[Name] Hayao Suzuki(鈴木 駿)
\item[\xout{Twitter} X] \href{https://twitter.com/CardinalXaro}{@CardinalXaro}
\item[Work] Software Developer @ BeProud Inc.
\end{description}
\end{block}

\begin{center}
\begin{itemize}
\item 株式会社ビープラウド \includegraphics[width=3cm]{bplogo.png} 
\begin{itemize}
\item IT勉強会支援プラットフォーム \includegraphics[width=2cm]{connpass_logo_1.png}
\item Python独学プラットフォーム \includegraphics[width=1cm]{pyq_logo_color.png} 
\item システム開発ドキュメントサービス \includegraphics[width=3cm]{tracery.png} 
\end{itemize}
\end{itemize}
\end{center}

\end{frame}

\begin{frame}\frametitle{自己紹介}

\begin{block}{発表したトーク(抜粋)}
\begin{itemize}
\item \structure{SymPyによる数式処理}(PyCon JP 2018)
\item \structure{インメモリーストリーム活用術}(PyCon JP 2020)
\item \structure{組み込み関数powの知られざる進化}(PyCon JP 2021)
\item \structure{Let's implement useless Python objects}(PyCon APAC 2023)
\end{itemize}
\end{block}
\url{https://xaro.hatenablog.jp/}に一覧があります。
\end{frame}

\begin{frame}\frametitle{自己紹介}

\begin{block}{翻訳した本}
\begin{itemize}
\item \structure{Python Distilled}(O'Reilly Japan) \structure{本日の主役} 
\end{itemize}
\end{block}

\begin{block}{監訳した本}
\begin{itemize}
\item \structure{入門 Python 3 第2版}(O'Reilly Japan) 
\item \structure{ロバストPython}(O'Reilly Japan) 
\end{itemize}
\end{block}

\end{frame}

\section{Python Distilled}

\begin{frame}\frametitle{今日のテーマ}

\begin{center}
\includegraphics[width=5cm]{picture_large978-4-8144-0046-1.jpeg}
\end{center}

\end{frame}

\begin{frame}\frametitle{今日のテーマ}

\begin{center}
\includegraphics[width=5cm]{original.jpg}
\end{center}

\end{frame}

\begin{frame}\frametitle{Python Distilled}

\begin{block}{原著:Python Distilled}
\begin{description}
\item[著者] David M. Beazley
\item[出版年] 2021年9月
\item[出版社] Addison-Wesley(Pearson)
\end{description}
\end{block}

\begin{block}{邦訳:Python Distilled プログラミング言語Pythonのエッセンス}
\begin{description}
\item[訳者] 鈴木 駿
\item[出版年] 2023年10月
\item[出版社] オライリー・ジャパン
\end{description}
\end{block}

Pearsonとの契約上、邦訳の表紙は動物ではない。

\end{frame}

\begin{frame}\frametitle{翻訳の流れ}

\begin{block}{翻訳出版までの軌跡}
\begin{itemize}
\item 2022年4月 興味本位で原著電子版を購入
\item 2022年5月 オライリーの編集者に原著を紹介する(雑談レベル)
\item 2022年6月 翻訳版権取得に向けて動き出す
\item 2022年7月 翻訳版権取得、翻訳の打診、翻訳に挑戦しようと決意
\item 2022年9月 翻訳を開始する(ロバストPythonの監訳と並行)
\item 2023年4月 一通り翻訳が完了、推敲の日々
\item 2023年9月 翻訳作業完了
\end{itemize}
\end{block}

\end{frame}

\begin{frame}\frametitle{Python Distilledってどんな本?}

\begin{block}{原著者「はじめに」より}
この『Python Distilled』はPythonによるプログラミングについての書籍です。
Pythonで可能なことや、あるいは行われたことをすべて文書化しようというわけではありません。
本書の目的は、現代的であり厳選、つまり蒸留(distilled)されたプログラミング言語Pythonの核心を紹介することです。
(中略)
しかし、それはまた、ソフトウェアライブラリを書き、Pythonの何たるかを知り、何が最も役に立つかを見出した結果でもあるのです。
\end{block}

\end{frame}

\begin{frame}\frametitle{つまり、どんな本?}

\begin{block}{一言でまとめると}
プログラミング言語Pythonそのものに特化した本
\end{block}

\end{frame}

\begin{frame}\frametitle{書評}

\begin{block}{\xout{Twitter} Xで見かけた書評}
\href{https://twitter.com/jeanmarcalkazzi/status/1512161719585583112}{No human should be allowed to write Python code before reading it.}
\end{block}

\begin{block}{渋川さんによる書評}
\href{https://future-architect.github.io/articles/20231116a/}{着実にPythonを自らの血肉にしていきたい人向けの本です。}
\end{block}

\end{frame}

\begin{frame}\frametitle{Pythonの学び方}

\begin{center}
\Huge{君たちはどうPythonを学ぶか}
\end{center}

\end{frame}

\begin{frame}\frametitle{Pythonの学び方}

\begin{center}
\Huge{\url{https://docs.python.org/ja/3/}}
\end{center}

\end{frame}

\begin{frame}\frametitle{Pythonの学び方}

\begin{block}{Pythonは公式ドキュメントが充実}
\begin{itemize}
\item \url{https://docs.python.org/ja/3/}
\end{itemize}
\end{block}

\begin{alertblock}{アレはどこに書いてあるの?}
そうそう、アレだよ、アレ、あそこにあるよ。
\end{alertblock}
発表者は親の影響で中日ファンでしたが、最近はまったく野球を見ていません。

\end{frame}

\begin{frame}\frametitle{突然のクイズ}

\begin{center}
\Huge{Pythonのアレ、どこに書いてあるかなクイズ!}
\end{center}

\end{frame}

\begin{frame}\frametitle{第1問}

\begin{block}{問題:シングルトンの比較}
\texttt{None}や\texttt{True}、\texttt{False}はシングルトンであり、
シングルトンは\texttt{is}文で比較します。
この注意事項はドキュメントのどこに書かれているでしょうか?
\end{block}

\only<2->{

\begin{exampleblock}{解答:2箇所}
\begin{itemize}
\item \href{https://docs.python.org/ja/3/reference/expressions.html}{言語仕様}(注意喚起)
\item \href{https://peps.python.org/pep-0008/}{PEP 8}(簡単な理由も)
\end{itemize}
\end{exampleblock}
実質的にはPEP 8だけとも言える。

}

\end{frame}

\begin{frame}\frametitle{第2問}

\begin{block}{問題:関数のデフォルト引数}
関数のデフォルト引数を使う際はイミュータブルなオブジェクトを使います。
この注意事項はドキュメントのどこに書かれているでしょうか?
\end{block}

\only<2->{

\begin{exampleblock}{解答:2箇所}
\begin{itemize}
\item \href{https://docs.python.org/ja/3/tutorial/controlflow.html}{チュートリアル}(注意喚起)
\item \href{https://docs.python.org/ja/3/faq/programming.html}{プログラミングFAQ}(デフォルト引数の仕組みについて)
\end{itemize}
\end{exampleblock}

}

\end{frame}

\begin{frame}\frametitle{第3問}

\begin{block}{問題:\texttt{with}文}
Python 2.5から\texttt{with}文が導入されました。
\texttt{with}文の使い方はどこに書かれているでしょうか?
\end{block}

\only<2->{

\begin{exampleblock}{解答:3箇所}
\begin{itemize}
\item \href{https://docs.python.org/ja/3/tutorial/inputoutput.html}{チュートリアル}(存在を示唆するだけ)
\item \href{https://docs.python.org/ja/3/reference/compound_stmts.html}{言語リファレンス}(\texttt{with}文の構文とコンテキストマネージャについて)
\item \href{https://peps.python.org/pep-0343/}{PEP 343}(\texttt{with}の導入経緯や背景について)
\end{itemize}
\end{exampleblock}

}

\end{frame}

\begin{frame}\frametitle{第4問}

\begin{block}{問題:\texttt{\_\_init\_\_()}と\texttt{\_\_new\_\_()}}
クラスのインスタンスを実際に生成するのは\texttt{\_\_new\_\_()}、
インスタンスの初期化は\texttt{\_\_init\_\_()}です。
この関係について書かれているのはどこでしょうか?
\end{block}

\only<2->{

\begin{exampleblock}{解答:1箇所}
\begin{itemize}
\item \href{https://docs.python.org/ja/3/reference/datamodel.html}{言語リファレンス}
\end{itemize}
\end{exampleblock}
\texttt{\_\_init\_\_()}はコンストラクタじゃないよ!
}

\end{frame}

\begin{frame}\frametitle{第5問}

\begin{block}{問題:\texttt{from module import *}}
\texttt{from module import *}が可能なのはモジュールレベルのインポートで、
クラスや関数内部ではできません。
この事実について書かれているのはどこでしょうか?
\end{block}

\only<2->{

\begin{exampleblock}{解答:1箇所}
\begin{itemize}
\item \href{https://docs.python.org/ja/3/reference/simple_stmts.html}{言語リファレンス}
\end{itemize}
\end{exampleblock}
ただし、\texttt{from module import *}は使うなと\href{https://docs.python.org/ja/3/faq/programming.html}{注意喚起}されている

}

\end{frame}

\begin{frame}\frametitle{Pythonの学び方}

\begin{block}{Pythonは公式ドキュメントが充実}
\begin{itemize}
\item 大体公式ドキュメントやPEPに書かれている
\item チュートリアルと標準ライブラリだけで何とかなる
\end{itemize}
\end{block}

\begin{alertblock}{公式ドキュメントは膨大すぎる}
\begin{itemize}
\item 突っ込んだ内容だと言語リファレンスやPEPを探ることになる
\item 言語リファレンスは「そっけない書き方」、読み物的に読めない。
\end{itemize}
\end{alertblock}

\end{frame}

\begin{frame}\frametitle{Pythonの学び方}

\begin{block}{書籍から学ぶ:入門書}
\begin{itemize}
\item 基本的には初心者向き、突っ込んだ内容には触れていない
\item 筆者の工夫として取捨選択が行われている
\end{itemize}
\end{block}

\begin{exampleblock}{入門書の具体例}
\begin{itemize}
\item 『入門 Python 3 第2版』(800ページ)、入門部分は270ページ
\item 『Pythonチュートリアル 第4版』(264ページ)、底本は公式ドキュメント
\end{itemize}
\end{exampleblock}

\end{frame}

\begin{frame}\frametitle{Pythonの学び方}

\begin{block}{書籍から学ぶ:(比較的)高いレベルの本}
\begin{itemize}
\item Python言語を網羅しようとしている
\item 必然的に分厚くなり、読み通すのが大変
\end{itemize}
\end{block}

\begin{exampleblock}{高いレベルの具体例}
\begin{itemize}
\item 『初めてのPython 第3版』(808ページ)、原書第5版は1648ページ
\item 『Fluent Python』(832ページ)、原書第2版は983ページ
\end{itemize}
\end{exampleblock}

\end{frame}

%\begin{frame}\frametitle{Pythonの学び方}
%
%\begin{block}{コンパクトにまとまった本}
%\begin{itemize}
%\item トピックごとに整理するか、テーマを決めるか
%\item 網羅性はやや低いが、読みやすい
%\end{itemize}
%\end{block}
%
%\begin{exampleblock}{コンパクトにまとまった本の具体例}
%\begin{itemize}
%\item 『Effective Python 第2版』(456ページ)、90項目に整理
%\item 『ロバストPython』(384ページ)、型ヒントが中心
%\end{itemize}
%\end{exampleblock}
%
%\end{frame}

\begin{frame}\frametitle{Pythonの学び方}

\begin{block}{インターネット、または検索で探す}
\begin{itemize}
\item 玉石混合
\item 結局は公式ドキュメントに落ち着く
\item 調べたいことがわかっていないと使えない
\end{itemize}
\end{block}

\begin{exampleblock}{品質が高いWeb資料}
\begin{itemize}
\item 『\href{https://pycamp.pycon.jp/textbook/index.html}{Python Boot Camp Text}』初心者向けチュートリアルイベントの資料
\item 『\href{https://pythontutor.com/}{Python Tutor}』Pythonの動きを視覚的に確認できるサイト
\end{itemize}
\end{exampleblock}

\end{frame}

\begin{frame}\frametitle{Pythonの学び方}

\begin{block}{最近の流行り:GPTsに聞いてみる}
\begin{itemize}
\item 俺たちのChatGPT先生
\item 調べたいことがわかっていないと使えない
\item たたき台として最適
\item ある程度Pythonをわかっていないと使いこなせない(私見)
\end{itemize}
\end{block}

\begin{exampleblock}{エレミヤ 14:14(聖書協会共同訳より)}
主は私に言われた。
「預言者たちは、私の名によって偽りの預言をしている。私は彼らを遣わしたこともなく、彼らに命じたこともなく、彼らに語ったこともない。彼らは偽りの幻と空しい占いと自分の心の欺きをあなたがたに預言しているのだ。
\end{exampleblock}

\end{frame}

\begin{frame}\frametitle{Pythonの学び方}

\begin{block}{Python Distilled}
\begin{itemize}
\item Python言語そのものに特化した本
\item 言語リファレンスに書いてあることが336ページにまとまっている
\end{itemize}
\end{block}

\begin{alertblock}{Python Distilledに書いていないこと}
\begin{itemize}
\item 型ヒント周り(『ロバストPython』読もう)
\item 非同期処理
\item 3rdパーティライブラリ、エコシステム周り
\end{itemize}
\end{alertblock}

\end{frame}

\begin{frame}\frametitle{FAQ}

\begin{exampleblock}{本当によくある質問}
『Python Distilled』と『入門 Python 3 第2版』の違いは?
\end{exampleblock}

\begin{block}{Python Distilledに書いていないこと}
\begin{itemize}
\item 『入門 Python 3 第2版』はエコシステムもしっかり触れている
\item 『Python Distilled』は言語コアに特化した本
\end{itemize}
\end{block}

\end{frame}

\begin{frame}\frametitle{Python Distilled}

\begin{center}
\Huge{結局、何が書いてあるの?}
\end{center}

\end{frame}

\begin{frame}\frametitle{1章 Pythonの基礎}

\begin{block}{1章 Pythonの基礎}
\begin{itemize}
\item 変数やデータ型、式、制御構造、関数、クラス、入出力についての概説
\item 原書はPython 3.8以降を想定、翻訳では3.11まで対応できるようにした
\item わかる人は飛ばしても大丈夫
\end{itemize}
\end{block}

\end{frame}

\begin{frame}\frametitle{1章 Pythonの基礎}

\begin{exampleblock}{例:スタックベースの計算機}
大半の場合、継承は最良の解決策ではありません。
例えば、単純なスタックベースの計算機を作りたいとします。
\end{exampleblock}

\begin{alertblock}{翻訳時に抱いた懸念}
\begin{itemize}
\item 「スタックベースの計算機」は説明なしで通じるのか
\item 逆ポーランド記法の説明を追加するべきか
\end{itemize}
\end{alertblock}

\end{frame}

\begin{frame}\frametitle{2章 演算子、式、データ操作}

\begin{block}{2章 演算子、式、データ操作}
\begin{itemize}
\item 式、演算子、評価規則について
\item 基本的なデータ構造についても説明
\end{itemize}
\end{block}

\end{frame}

\begin{frame}[fragile]\frametitle{2章 演算子、式、データ操作}

\begin{alertblock}{暗黙的な真偽値評価の罠}
\begin{pygments}{python}
def f(x, items=None):
    if not items:
        items = []
    items.append(x)
    return items
\end{pygments}
\end{alertblock}

\end{frame}

\begin{frame}[fragile]\frametitle{2章 演算子、式、データ操作}

\begin{exampleblock}{実行例}
\begin{pygments}{python}    
>>> f(4)
[4]
>>> a = []
>>> f(3, a)
[3]
>>> a  # 更新されない!
[]
\end{pygments}
\end{exampleblock}

\end{frame}

\begin{frame}\frametitle{3章 プログラムの構造と制御構造}

\begin{block}{3章 プログラムの構造と制御構造}
\begin{itemize}
\item 条件分岐、ループ、例外、コンテキストマネージャについて
\item 「3.4 例外」は例外なく読もう!
\end{itemize}
\end{block}

\end{frame}

\begin{frame}[fragile]\frametitle{3章 プログラムの構造と制御構造}

\begin{exampleblock}{例:例外値を保持する変数の生存範囲}
\begin{pygments}{python}    
try:
    int("N/A")
except ValueError as e:
    print("Failed:", e)
print(e)  # NameError
\end{pygments}
\end{exampleblock}

\end{frame}

\begin{frame}[fragile]\frametitle{3章 プログラムの構造と制御構造}

\begin{exampleblock}{例:例外の連鎖}
\begin{pygments}{python}    
try:
    x = int("N/A")
except Exception as e:
    raise ApplicationError("It failed") from e
\end{pygments}
\end{exampleblock}

\begin{exampleblock}{例:想定外の連鎖}
\begin{pygments}{python}    
try:
    x = int("N/A")
except Exception as e:
    print("It failed:", err)  # NameError
\end{pygments}
\end{exampleblock}

\end{frame}

\begin{frame}\frametitle{3章 プログラムの構造と制御構造}

\begin{block}{\texttt{\_\_cause\_\_}属性と\texttt{\_\_context\_\_}属性}
\begin{itemize}
\item \texttt{\_\_cause\_\_}属性は意図して例外を連鎖した時に参照する
\item \texttt{\_\_context\_\_}属性は例外処理中の想定外の例外発生時の情報源
\end{itemize}
\end{block}

\end{frame}

\begin{frame}\frametitle{4章 オブジェクト、型、プロトコル}

\begin{block}{4章 オブジェクト、型、プロトコル}
\begin{itemize}
\item Pythonの基本的なオブジェクトモデルとメカニズムについて
\item PyCon APAC 2023の発表の元ネタの1つ
\end{itemize}
\end{block}

\end{frame}

\begin{frame}[fragile]\frametitle{4章 オブジェクト、型、プロトコル}

\begin{exampleblock}{例:参照カウント}
\begin{pygments}{python}    
>>> a = 37  # 値37を持つオブジェクトを作成する
>>> b = a  # オブジェクトの参照カウント増加
>>> c = []
>>> c.append(b)  # オブジェクトの参照カウント増加
\end{pygments}
\end{exampleblock}

\end{frame}

\begin{frame}[fragile]\frametitle{4章 オブジェクト、型、プロトコル}

\begin{alertblock}{例:参照とコピー}
\begin{pygments}{python}    
>>> a = [1, 2, 3, 4]
>>> b = a  # bはaの参照
>>> b is a
True
>>> b[2] = -100  # bの要素を変更する
>>> a  # aの要素も変更される
[1, 2, -100, 4]
\end{pygments}
\end{alertblock}

\end{frame}

\begin{frame}[fragile]\frametitle{4章 オブジェクト、型、プロトコル}

\begin{exampleblock}{例:整数型と浮動小数点数型}
\begin{pygments}{python}    
>>> a = 42
>>> b = 3.7
>>> a.__add__(b)
NotImplemented
>>> b.__radd__(a)
45.7
\end{pygments}
\end{exampleblock}

\end{frame}

\begin{frame}\frametitle{5章 関数}

\begin{block}{5章 関数}
\begin{itemize}
\item 関数定義、適用、スコープ、クロージャ、デコレータ、関数型プログラミング
\item 5.16節と5.17節のコールバック関数に関する説明は内容が濃い
\end{itemize}
\end{block}

\end{frame}

\begin{frame}[fragile]\frametitle{5章 関数}

\begin{exampleblock}{例:動的な関数生成}
\begin{pygments}{python}    
def make_init(*names):
    params = ", ".join(names)
    code = f"def __init__(self, {params}):\n"
    for name in names:
        code += f"    self.{name} = {name}\n"
    d = {}
    exec(code, d)
    return d["__init__"]
\end{pygments}
\end{exampleblock}
\texttt{NamedTuple}や\texttt{@dataclass}で活用されているテクニック

\end{frame}

\begin{frame}\frametitle{6章 ジェネレータ}

\begin{block}{6章 ジェネレータ}
\begin{itemize}
\item ジェネレータの基礎から応用まで
\item 「コルーチン」の歴史も
\end{itemize}
\end{block}

\end{frame}

\begin{frame}[fragile]\frametitle{6章 ジェネレータ}

\begin{exampleblock}{例:ジェネレータの委譲}
\begin{pygments}{python}    
def flatten(items):
    for i in items:
        if isinstance(i, list):
            yield from flatten(i)
        else:
            yield i

a = [1, 2, [3, [4, 5], 6, 7], 8]
for x in flatten(a):
    print(x, end=" ")
\end{pygments}
\end{exampleblock}

\end{frame}

\begin{frame}[fragile]\frametitle{6章 ジェネレータ}

\begin{exampleblock}{例:拡張ジェネレータ}
\begin{pygments}{python}    
def receiver():
    print("Ready to receive")
    while True:
        n = yield
        print("Got", n)

>>> r = receiver()
>>> r.send(None)
Ready to receive
>>> r.send("Hello")
Got Hello
>>> r.close()
\end{pygments}
\end{exampleblock}

\end{frame}

\begin{frame}\frametitle{7章 クラス}

\begin{block}{7章 クラス}
\begin{itemize}
\item クラスに関する概念をトップダウンで学ぶ
\item 相当マニアック
\end{itemize}
\end{block}

\end{frame}

\begin{frame}\frametitle{8章 モジュールとパッケージ}

\begin{block}{8章 モジュールとパッケージ}
\begin{itemize}
\item モジュールの概念が焦点
\item パッケージングについては触れていない
\end{itemize}
\end{block}

\end{frame}

\begin{frame}\frametitle{9章 入力と出力}

\begin{block}{9章 入力と出力}
\begin{itemize}
\item I/O処理に必要なテクニックと抽象化
\item 後半のI/Oライブラリの概要は流し読みでOK
\end{itemize}
\end{block}

\end{frame}

\begin{frame}\frametitle{10章 組み込み関数と標準ライブラリ}

\begin{block}{10章 組み込み関数と標準ライブラリ}
\begin{itemize}
\item 組み込み関数と標準ライブラリ、組み込み例外の一覧
\item 流し読みでOK
\end{itemize}
\end{block}

\end{frame}

\begin{frame}\frametitle{まとめ}

\begin{block}{さっそく購入しよう}
\begin{itemize}
\item \url{https://www.oreilly.co.jp/books/9784814400461/}
\item \url{https://www.ohmsha.co.jp/book/9784814400461/}
\end{itemize}
\end{block}

\begin{block}{オライリー学習プラットフォームとは}
\begin{itemize}
\item \url{https://www.oreilly.co.jp/online-learning/}
\item 6万冊以上の書籍と3万時間以上の動画(日本語もある!)
\item 業界エキスパートによるライブイベント
\item インタラクティブなシナリオとサンドボックスを使った実践的な学習
\item 公式認定試験対策資料
\item 『Python Distilled』もオライリー学習プラットフォームで読み放題(すごい)
\end{itemize}
\end{block}

\end{frame}

\end{document}